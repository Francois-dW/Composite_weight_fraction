% Appendix Reference for Composite Material Analysis Tool
% Copy this into your Overleaf project appendix

\section{Composite Material Analysis Tool}

The calculations and analysis presented in this work were performed using a custom-developed Python-based Composite Material Analysis Tool. This software implements theoretical models for fiber-reinforced composite materials, including the Halpin-Tsai equations and rule of mixtures approaches.

\subsection{Key Features}

The tool provides:
\begin{itemize}
    \item Calculation of composite properties based on constituent materials (fiber and matrix)
    \item Automatic determination of analysis cases (excess matrix vs. fiber saturation)
    \item Porosity effects modeling with efficiency exponents
    \item Fiber length and orientation efficiency factor calculations
    \item Multi-material comparison capabilities
    \item Experimental data integration with curve fitting optimization
    \item Parameter estimation from experimental measurements
\end{itemize}

\subsection{Theoretical Background}

The composite stiffness is calculated using:
\begin{equation}
E_c = (\eta_0 \eta_1 V_f E_f + V_m E_m)(1 - V_p)^n
\end{equation}

where $E_c$ is the composite stiffness, $\eta_0$ is the fiber orientation efficiency factor, $\eta_1$ is the fiber length efficiency factor, $V_f$ is the fiber volume fraction, $E_f$ is the fiber stiffness, $V_m$ is the matrix volume fraction, $E_m$ is the matrix stiffness, $V_p$ is the porosity volume fraction, and $n$ is the porosity efficiency exponent.

The fiber length efficiency factor is calculated according to:
\begin{equation}
\eta_1 = 1 - \frac{\tanh(\beta L/2)}{\beta L/2}
\end{equation}

where $\beta = \frac{2G_m}{E_f \ln(\kappa / \text{aspect ratio})}$, $G_m$ is the matrix shear modulus, $L$ is the fiber length, and $\kappa$ is a packing parameter.

\subsection{Software Availability}

The Composite Material Analysis Tool is available as open-source software:

\begin{itemize}
    \item \textbf{Repository:} \url{https://github.com/Francois-dW/Composite_weight_fraction}
    \item \textbf{Version:} November 2025 (with curve fitting optimization)
    \item \textbf{License:} Educational and research purposes
    \item \textbf{Platform:} Python 3.8+ with NumPy, Matplotlib, and SciPy
    \item \textbf{Executable:} Standalone Windows executable available (73.8 MB)
\end{itemize}

The software includes comprehensive documentation, validation test cases, and example experimental data files. All calculations can be reproduced using the provided source code and input parameters documented in this work.

% BibTeX entry for references section:
% Copy this into your .bib file

% @misc{composite_analysis_tool_2025,
%   author       = {de Wet, François},
%   title        = {Composite Material Analysis Tool},
%   year         = {2025},
%   publisher    = {GitHub},
%   howpublished = {\url{https://github.com/Francois-dW/Composite_weight_fraction}},
%   note         = {Python-based tool for fiber-reinforced composite analysis with experimental data integration}
% }

% In-text citation example:
% All composite property calculations were performed using the Composite Material Analysis Tool \cite{composite_analysis_tool_2025}.
